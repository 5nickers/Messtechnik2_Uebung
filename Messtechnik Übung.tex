\input{header.tex}


\begin{document}

\maketitle

Dieser Text ist unter dieser \href{http://creativecommons.org/licenses/by-nc-sa/4.0/}{Creative Commons} Lizenz veröffentlicht.

\textcolor{red}{Ich erhebe keinen Anspruch auf Vollständigkeit oder Richtigkeit. Falls ihr Fehler findet oder etwas fehlt, dann meldet euch bitte über den Emailkontakt.}

\tableofcontents


\newpage

\section{Aufgabe 2.1}

\subsection*{a)}

\begin{align*}
	&\text{linearer Mittelwert:} \qquad \bar{u} = \frac{1}{T} \int u \d t = \unit[0]{V} \\
	&\text{Gleichricht Mittelwert:} \qquad \bar{|u|} = \frac{1}{T} \int |u| \d t = \unit[1]{V} \\
	&\text{effektiver Mittelwert:} \qquad u_{eff} = U = \sqrt{\frac{1}{T} \cdot \int u^2 \d t} = \left(\frac{1}{T} \cdot \left( 1^2 \cdot V^2 \cdot \frac{T}{2} + \left(- \unit[1]{V} \right)^2 \cdot \frac{T}{2} \right) \right)^{0,5} = \unit[1]{V} \\
	\hfill \\
	&\text{Formfaktor:} \qquad F = \frac{U}{|u|} = \frac{\unit{1}[V]}{\unit{1}[V]} = \unit{1}[V]
\end{align*}

\subsection*{b)}

\begin{align*}
	&\text{linearer Mittelwert:} \qquad \bar{u} = \frac{1}{T} \int u \d t = \frac{1}{T} \left(2 \cdot \frac{T}{T} - 1 \cdot \frac{T}{T} \right) = \unit[0,5]{V} \\
	&\text{Gleichricht Mittelwert:} \qquad \bar{|u|} = \frac{1,5}{T} \int |u| \d t = = \frac{1}{T} \left(2 \cdot \frac{T}{T} + 1 \cdot \frac{T}{T} \right) \unit[1]{V} \\
	&\text{effektiver Mittelwert:} \qquad u_{eff} = U = \sqrt{\frac{1}{T} \cdot \int u^2 \d t} = \left(\frac{1}{T} \left(2^2 \cdot \frac{T}{T} + 1^2 \cdot \frac{T}{T} \right) \right)^{0,5} = \unit[1,58]{V} \\
	\hfill \\
	&\text{Formfaktor:} \qquad F = \frac{U}{|u|} = \frac{\unit{1,58}[V]}{\unit{1,5}[V]} = \unit{1,05333}[V]
\end{align*}


\subsection*{c)}

Wir betrachten den Sinus hier als Sinus von x statt von t, da wir dann nicht substituieren müssen.

\begin{align*}
	&\text{linearer Mittelwert:} \qquad \bar{u} = \frac{1}{2 \pi} \int_0^\pi \overset{\wedge}{u} \sin(x) \d x = \overset{\wedge}{u} \cdot \frac{1}{2 \pi} \left[ - \cos(x) \right]_0^\pi = \unit[3,18]{V} \\
	&\text{Gleichricht Mittelwert: Das Signal ist schon gleichgerichtet, deshalb gilt $\bar{u} = |\bar{u}|$}  \\
	&\text{effektiver Mittelwert:} \qquad u_{eff} = U = \sqrt{ \frac{1}{2 \pi} \int_0^\pi \overset{\wedge}{u}^2\sin(x)^2 \d x} = \overset{\wedge}{u} \sqrt{\left[ \frac{1}{2 \pi} \cdot \frac{x - cos(x) \cdot \sin(x)}{2}\right]_0^\pi} = \unit[5]{V} \\
	\hfill \\
	&\text{Formfaktor:} \qquad F = \frac{U}{|u|} = \frac{\unit{5}[V]}{\unit{3,18}[V]} = \unit{1,57}[V]
\end{align*}


\newpage

\subsection*{d)}

Wir müssen in diesem Fall nur bis $\frac{\pi}{2}$ betrachten, weil es sich ab dann schon wiederholt:

\begin{align*}
	&\text{linearer Mittelwert:} \qquad \bar{u} = \frac{2}{\pi} \int_0^\pi \overset{\wedge}{u} \sin(x) \d x = \overset{\wedge}{u} \cdot \frac{2}{\pi} \left[ - \cos(x) \right]_0^\pi = \unit[6,37]{V} \\
	&\text{Gleichricht Mittelwert: Das Signal ist schon gleichgerichtet, deshalb gilt $\bar{u} = |\bar{u}|$}  \\
	&\text{effektiver Mittelwert:} \qquad u_{eff} = U = \sqrt{ \frac{2}{\pi} \int_0^\pi \overset{\wedge}{u}^2\sin(x)^2 \d x} = \overset{\wedge}{u} \sqrt{\left[ \frac{2}{\pi} \cdot \frac{x - cos(x) \cdot \sin(x)}{2}\right]_0^\pi} = \unit[7,07]{V} \\
	\hfill \\
	&\text{Formfaktor:} \qquad F = \frac{U}{|u|} = \frac{\unit{7,07}[V]}{\unit{6,37}[V]} = \unit{1,11}[V]
\end{align*}


\section{Aufgabe 3.1}


\subsection*{a)}

Der Strom durch den Shuntwiderstand $R_S$ heißt $I_S$:

\begin{align*}
R_S &= 90 + 9 + 0,9 + 0,1 = \unit[100]{\Omega} \\
\frac{I_{max}}{I_S} &= \frac{100}{400} \\
\Leftrightarrow I_{max} &= \frac{1}{4} \cdot I_S 
\intertext{Wir wissen das gilt:}
I_{ges} = I_{max} + I_S &= \unit[1]{mA} \Leftrightarrow I_S = \unit[0,8]{mA} \Rightarrow I_{max} = \unit[0,2]{mA}
\end{align*}



\subsection*{b)}

\begin{align*}
\frac{I_{max}}{I_S} &= \frac{0,1}{400 + 99,9} \approx \frac{0,1}{500} \\
\Leftrightarrow I_{max} &= \frac{0,1}{500} \cdot I_S = \frac{I_{ges}}{5000} = \unit[0,2]{mA}
\end{align*}


\newpage

\subsection*{c)}

\begin{figure}[h]
	\centering
	\includegraphics[scale=0.2]{A3_1_1.jpg}
\end{figure}



\begin{align*}
R_{ges} &= 920 + \frac{400 \cdot 100}{400 + 100} = \unit[1000]{\Omega}
\end{align*}


Von dem $\unit[920]{\Omega}$ fließen also \unit[1]{mA} nach links splittet sich dann wie im Aufgabenteil zuvor auf und wir erhalten dann wieder ein $I_{max} = \unit[0,2]{mA}$.

Bei einem $R_{ges} = \unit[100]{k \Omega}$ und $\unit[100]{V}$ haben wir genau die selbe Situation nur andere Werte.


\subsection*{d)}

Die Innenwiderstände sind wie folgt:

\begin{tabular}{|l|l|}
	\hline $\unit[100]{V}$ Messbereich & $\unit[100]{k\Omega}$  \\ 
	\hline $\unit[1]{V}$ Messbereich & $\unit[1]{k\Omega}$ \\ 
	\hline $\unit[1]{mA}$ Messbereich & $\unit[80]{\Omega}$ \\ 
	\hline $\unit[1]{A}$ Messbereich & $\unit[0,1]{\Omega}$ \\ 
	\hline 
\end{tabular} 

\section{Aufgabe 3.2}

Diese Aufgabe wurde nicht gemacht weil sie veraltet ist.

\newpage


\section{Aufgabe 3.3}

\begin{figure}[h]
	\centering
	\includegraphics[scale=0.15]{A3_3_1.jpg}
\end{figure}




\begin{align*}
U &= \left( R_V + R_U \right) \cdot I_{U,max} \\
\Leftrightarrow R_V &= \frac{U - R_U \cdot I_{u,max}}{I_{u,max}} = \frac{U}{I_{u,max}} \cdot R_U = \frac{100}{0,1} \cdot 400 = \unit[1]{M \Omega}
\intertext{Nun bestimmen wir noch den Shuntwiderstand:}
\frac{I_S}{I_{I,max}} &= \frac{R_I}{R_S} \Leftrightarrow I_S = \frac{R_I}{R_S} \cdot I_{I,max} \\
I &= I_S + I_{I,max} = I_{I,max} \cdot \left( \frac{R_I}{R_S} + 1 \right) \\
R_S = R_I \cdot \frac{I_{I,max}}{I - I_{I,max}} &= \frac{\unit[10]{mA}}{\unit[10]{A} - \unit[10]{mA}} \cdot \unit[10]{\Omega} = \unit[0,01]{\Omega}
\end{align*}



\section{Aufgabe 4.1}


\subsection*{a)}

Wir berechnen zunächst den Widerstand $R_A$ bei der Strommessung:

\begin{align*}
R_A &= \frac{\unit[200]{mV}}{\unit[20]{A}} = \unit[0,01]{\Omega}
\intertext{Zur Abschätzung des Innenwiderstands $R_Q$ tun wir so als wäre die Messung korrekt:}
R_Q = \frac{U_V}{I_A} &= \frac{1,2}{13,33} = \unit[0,09]{\Omega}
\intertext{Der Widerstand ist also um etliche Größenordnungen kleiner als $R_V$ und daher können wir die Spannungsmessung als genau annehmen $U_0 = U_V$. Nun bestimmen wir den Widerstand bei der Strommessung:}
I_A &= \frac{U_0}{R_Q + R_A} \\
\Leftrightarrow U_0 &= I_A \cdot R_Q + I_A \cdot R_A \\
\Leftrightarrow R_Q = \frac{U_0 - I_A \cdot R_A}{I_A} &= \frac{1,2 - 13,33 \cdot 0,01}{13,33} = \unit[0,08]{\Omega}
\end{align*}

\subsection*{b)}

Es macht keinen Sinn bei der Spannung einen Fehler zu berechnen, weil der Fehler auf Grund des großen Widerstandes weit kleiner ist als das Messgerät anzeigen kann.

Für den Strom können wir den Fehler so berechnen:

\begin{align*}
I_K &= \frac{U_0}{R_Q} = \frac{1,2}{0,08} = \unit[15]{\Omega} \\
e &= I_A - I_K = \unit[- 1,67]{A} \\
e^* &= \frac{e}{I_K} = \frac{-1.67}{15} = -0,111 = \unit[-11,1]{\%}
\end{align*}


\section{Aufgabe 4.2}

Der Widerstand $R_h$ ist ein \textbf{einstellbarer} Widerstand, da mit diesem Widerstand kompensiert werden soll. Wir bestimmen zunächst de minimalen und den maximalen Strom:

\begin{align*}
I_{K,max} &= \frac{\unit[5]{V}}{\unit[50]{\Omega}} = \unit[0,1]{A} \\
I_{K,min} &= \frac{\unit[5]{V}}{\unit[200]{\Omega}} = \unit[0,025]{A}
\intertext{Die zugehörigen minimalen und maximalen Widerstände sind dann:}
R_{h,max} &= \frac{\unit[10]{V}}{\unit[0,025]{A}} = \unit[400]{\Omega} \\
R_{h,min} &= \frac{\unit[10]{V}}{\unit[0,1]{A}} = \unit[100]{\Omega}
\intertext{Die Genauigkeit muss sich als an dem kleineren Widerstand orientieren:}
\Delta R_{h,min} &= 100 \cdot 0,001 = \unit[0,1]{\Omega}
\intertext{Jetzt stellt sich die Frage wie sich die Spannung an dem Widerstand $R_S$ ändert, wenn wir den Wert von $R_h$ um eine Einheit aus der Kompensationsstellung verstellen:}
U_M &= (I - I_h) \cdot R_S = I \cdot R_S - \frac{U_h}{R_h} \cdot R_S
\intertext{Dazu müssen wir die obige Formel ableiten:}
\frac{\p U_M}{\p R_h} &= \frac{U_h}{R_h^2} \cdot R_S \\
\Rightarrow \Delta U_M &= \frac{U_h \cdot R_S}{R_h^2} \cdot \Delta R_h = \frac{\unit[10]{V} \cdot \unit[10^3]{\Omega}}{\left( \unit[100]{\Omega} \right)^2} \cdot \unit[0,1]{\Omega} = \unit[0,1]{V}
\end{align*}


\section{Aufgabe 4.3}

Diese Aufgabe ist nicht klausurrelevant und wurde noch nicht in der Übung besprochen.


\section{Aufgabe 4.4}

\begin{figure}[h]
	\centering
	\includegraphics[scale=0.1]{A4_4_1.jpg}
\end{figure}


\subsection*{a)}

Wir wollen die Empfindlichkeit der Messbrücke bestimmen. Die Empfindlichkeit hängt dabei folgendermaßen von der Spannungsänderung und der Dehnung ab:

\begin{align*}
E &= \frac{\Delta U_M}{\Delta \epsilon} = ?
\intertext{Zunächst drücken wir die Spannung $U_0$ mit den zwei möglichen Ästen ab:}
U_0 &= R_1 \cdot I_1 + R \cdot I_1 = 2 R T_1 \\
U_0 &= \left( R + \Delta R \right) \cdot I_2 + \left( R - \Delta R \right) \cdot I_2 = 2 R I_2 
\intertext{Nun könen wir nach dem Strom lösen:}
I = I_1 = I_2 &= \frac{U_0}{2 R} \\
\intertext{Wir betrachten den Maschenumlauf I:}
U_M &= I \cdot \left( R- \Delta R \right) - I \cdot R = 0 \\
\Leftrightarrow U_M &= I \cdot \Delta R = \frac{U_0}{2} \cdot \frac{\Delta R}{R}
\intertext{Das Verhältnis $\frac{\Delta R}{R}$ hängt direkt mit dem k-Faktor und der Dehnung zusammen:}
\frac{\Delta R}{R} &= k \cdot \epsilon \Rightarrow U_M = \frac{U_0}{2} \cdot k \cdot \epsilon 
\intertext{Die Empfindlichkeit ist die Ableitung dieser Gleichung nach der Dehnung:}
E &= \frac{\p U_M}{\p \epsilon} = \frac{U_0}{2} \cdot k = \frac{20}{2} \cdot 2 = \unit[20]{V}
\end{align*}

\subsection*{b)}

\begin{align*}
U_M &= \frac{20}{2} \cdot 2 \cdot 2 \cdot 10^{-4} = \unit[4]{mV}
\end{align*}


\section{Aufgabe 4.5}

Zunächst bestimmen wir die Impulsfrequenz:

\begin{align*}
f_s &= \frac{1000}{60} = \unit[16,7]{1/s}
\intertext{Damit der Fehler unter $\unit[1]{\%}$ liegt müssen wir mehr als 100 Impulse messen:}
T_{ref} &= \frac{100}{16,7} = \unit[5,98]{s}
\end{align*}

In der Praxis kann man sowas nicht verwenden, da die Totzeit viel zu groß ist.

\newpage

\section{Aufgabe 5.1}


\begin{figure}[h]
	\centering
	\includegraphics[scale=0.1]{A5_1_1.jpg}
\end{figure}


\subsection*{a)}

Wir machen zwei Umläufe, einmal im linken Aste und einmal über den Transistor im rechten:

\begin{align*}
\text{I:} \qquad 0 &= I_1 \cdot R_1 + U_{ein} - U_1  \\
\text{II:} \qquad 0 &= I_1 \cdot R_2 + U_2 - U_{ein} 
\intertext{Für $U_{ein}$ gilt zudem:}
U_2 &= - A \cdot U_{ein} \Leftrightarrow U_{ein} = - \frac{U_2}{A}
\intertext{Das setzen wir ein:}
\text{I:} \qquad 0 &= I_1 \cdot R_1 - \frac{U_2}{A} - U_1  \\
\text{II:} \qquad 0 &= I_1 \cdot R_2 + U_2 + \frac{U_2}{A} \\
\text{II:} \qquad I_1 &= - \frac{U_2}{R_2} \left( 1 + \frac{1}{A} \right)
\intertext{Wir setzen in die erste Gleichung ein:}
\text{I:} \qquad 0 &= U_2 \cdot \frac{R_1}{R_2} \left( 1 + \frac{1}{A} \right) - \frac{U_2}{A} - U_1 \\
\Leftrightarrow - U_1 &= U_2 \cdot \left[ \frac{R_1}{R_2} \left( 1 + \frac{1}{A} \right) + \frac{1}{A}  \right]
\intertext{Das Verhältnis ist dann:}
V^* &= \frac{U_2}{U_1} = - \frac{1}{\frac{R_1}{R_2} \left( 1 + \frac{1}{A} \right) + \frac{1}{A}} = \frac{A \cdot R_2}{R_1 \cdot A + R_1 + R_2}
\end{align*}

\subsection*{b)}

\begin{align*}
e^* &= \frac{V - V^*}{V^*} = \frac{V}{V^*} - 1 \\
&= \frac{R_2}{R_1} \cdot \frac{R_1 \cdot A + R_1 + R_2}{A \cdot R_2} - 1 = 1 + \frac{1}{A} + \frac{R_2}{R_1 \cdot A} - 1 = \frac{1}{A} + \frac{R_2}{R_1 \cdot A}
\end{align*}

\subsection*{c)}

Wir nehmen die Gleichung aus Teil b):

\begin{align*}
e^* &= \underbrace{\frac{1}{10^5}}_{\approx 0} + \frac{R_2}{R_1 \cdot 10^5} \leq 0,01
\end{align*} 

Damit das stimmt darf $\frac{R_2}{R_1}$ nicht größer als $10^3$ sein.


\section{Aufgabe 5.2}

\begin{figure}[h]
	\centering
	\includegraphics[scale=0.1]{A5_2_1.jpg}
\end{figure}


\subsection*{1)}

Der Widerstand $R_1$ bildet hier den Eingangswiderstand der Schaltung. Wenn der Fehler dabei maximal $\unit[1]{\%}$ sein soll, dann brauchen wir ein $R_1 = \unit[1000]{\Omega}$.

\subsection*{2)}

Statt den $\unit[2]{V}$ aus der Aufgabenstellung rechnen wir mit den von Prof. Helsper genannten $\unit[200]{mV}$.

\begin{align*}
|V| &= \frac{R_2}{R_1} = \frac{200}{20,78} = 18,55
\end{align*}

Es gilt damit $R_2 = \unit[18,55]{k \Omega}$

\subsection*{c)}

Die Zeitkonstante $\tau$ des Tiefpasses soll nun $T = \unit[1]{s}$ sein. Das können wir so bestimmen:

\begin{align*}
T &= R_2 \cdot C_2 \\
\Leftrightarrow C_2 &= \frac{T}{R_2} = \frac{1}{18550} = \unit[53,9]{\mu F}
\end{align*}


\section{Aufgabe 5.3}


\subsection*{a)}

Eine Verstärkung von $\unit[40]{dB}$ ist $V = 10^{40/20} = 10^2 = 100$. Die Verstärkung wir über die beiden Widerstände geregelt:

\begin{align*}
V &= 1 + \frac{R_2}{R_1}
\intertext{Es ist üblich den Widerstand $R_1$ genauso groß zu wählen wie den Innnenwiderstand der Spannungsquelle:}
100 &= 1 + \frac{R_2}{100} \\
\Leftrightarrow 99 &= \frac{R_2}{100} \\
\Leftrightarrow R_2 &= \unit[990]{k \Omega}
\end{align*}


\subsection*{b)}


Mit Gleichspannungsfehler ist hier die Wirkung der Offsetspannung gemeint. Wenn wir also eine Offsetspannung von $\unit[1]{mV}$ haben dann wird diese logischerweise mit der Verstärkung auf $\unit[100]{mV}$ verstärkt.


\subsection*{c)}

Wir berechnen zunächst die Grenzfrequenz, dabei wissen wir, das bei der Transitfrequenz die Verstärkung nur noch 1 ist.

\begin{align*}
f_g \cdot V &= f_T \cdot 1 \\
\Leftrightarrow f_g &= \frac{f_T}{100} = \frac{\unit[5]{MHz}}{100} = \unit[50]{kHz}
\intertext{Für den Fehler müssen wir uns erstmal klar machen was $\unit[-3]{dB}$ sind:}
10^{-3/20} &= 0,71
\intertext{Unsere Verstärkung ist also nur noch $V = V_0 \cdot 0,71$. Das können wir zur Fehlerberechnung nutzen:}
e^* = \frac{V_0 - V}{V_0} &= \frac{V_0 \cdot (1 - 0,71)}{V_0} = 0,29 \approx \unit[30]{\%}
\end{align*}



\section{Aufgabe 5.4}

Wurde nicht besprochen!


\section{Aufgabe 5.5}


\begin{figure}[h]
	\centering
	\includegraphics[scale=0.1]{A5_5_1.jpg}
\end{figure}


Wir dimensionieren $R_2$:

\begin{align*}
R_2 &= \frac{U_a}{I_{max}} = \frac{\unit[10]{V}}{\unit[10]{nA}} = \unit[1]{G \Omega} = \unit[10^9]{\Omega}
\end{align*}




\section{Aufgabe 5.6}

\subsection*{a)}

Wir berechnen zunächst den Widerstand und die Kapazität pro Scheibe:

\begin{align*}
R_S &= \frac{\rho \cdot d}{A} = \frac{10^{12} \cdot 0,5 \cdot 10^{-3}}{0,5 \cdot 10^{-4}} = \unit[10^{13}]{\Omega} \\
C_S &= \epsilon_0 \cdot \epsilon_r \cdot \frac{A}{d} = 8,85 \cdot 10^{-12} \cdot 5 \cdot \frac{0,5 \cdot 10^{-4}}{0,5 \cdot 10^{-3}} = \unit[4,43]{pF}
\intertext{Insgesamt haben wir dann:}
R_q = \frac{1}{5} \cdot R_S &= \frac{1}{5} \cdot 10^{13} = \unit[2 \cdot 10^{12}]{\Omega} \\
C_a = 5 \cdot C_S &= 5 \cdot 4,43 = \unit[22,13]{pF}
\end{align*}


\subsection*{b)}

\begin{align*}
Q_S &= k \cdot F = k \cdot A \cdot P \\
\Rightarrow Q &= 5 \cdot Q_S = 5 \cdot 2,3 \cdot 10^{-12} \cdot 0,5 \cdot 10^{-4} \cdot 10^6 = \unit[575]{pC}
\end{align*}


\subsection*{c)}

\begin{figure}[h]
	\centering
	\includegraphics[scale=0.15]{A5_6_1.jpg}
	\caption{oben: Schaltung für Teil c); unten: verbesserte Schaltung für Teil f)}
\end{figure}


\begin{align*}
C_{ges} &= C_a + C_k + C_v = 22 + 200 + 30 = \unit[252]{nF}
\intertext{Die Spannung um Quarz ist dann:}
U_q &= \frac{Q}{C_{ges}} = \frac{\unit[575]{pC}}{\unit[252]{pF}} = \unit[2,28]{V}
\intertext{Das Verhältnis von Quarz- zu Ausgangsspannung ist nun:}
V &= \frac{U_a}{U_q} = \frac{10}{2,28} = 4,38 
\intertext{Am Ausgang des Verstärkers sollte nicht mehr als $\unit[1]{mA}$ fließen. Da $R_1$ frei wählbar ist, wählen wir ihn so, das nur dieser Strom möglich ist. Wir erhalten dann $R_1 = \unit[10]{k \Omega}$. Für $R_2$ gilt dann:}
4,38 &= 1 + \frac{R_2}{R_1} \\
\Leftrightarrow R_2 &= (4,38 - 1) \cdot R_1 = \unit[33,8]{k \Omega}
\end{align*}


\subsection*{d)}

\begin{align*}
\tau &= C_{ges} \cdot R_{ges} 
\intertext{Wir nehmen an das $R_{ges} \approx R_k$ gilt:}
&\approx \unit[2,52]{pF} \cdot 10^{10} = \unit[2,52]{s}
\end{align*}


\subsection*{e)}

\begin{itemize}
	\item die Spannung hängt von der Kabelkapazität und Verstärkerkapazität ab
	\item fünf Scheiben $\rightarrow$ mehr Ladung, $\rightarrow$ nicht mehr Spannung
\end{itemize}


\subsection*{f)}

siehe Bild in Teil c).

Mit der Schaltung schließen wir alle Bauteile links von dem Verstärker kurz. Die Ladung landet dann auf dem Kondensator über dem Verstärker. Wir dimensionieren den Kondensator dann so:

\begin{align*}
U_a &= \left| - \frac{Q}{C} \right| \\
\Leftrightarrow C &= \frac{Q}{|U_a|} = \frac{\unit[575]{pC}}{\unit[10]{V}} = \unit[57,5]{pF}
\end{align*}


\subsection*{g)}

Die neue Zeitkonstante ist:

\begin{align*}
\tau &= R_C \cdot C = 10^{12} \cdot 57,5 = \unit[57,5]{s}
\end{align*}


\subsection*{h)}

\begin{align*}
U &= \frac{1}{C} \cdot \int i \d t 
\intertext{Wir differenzieren nun um die Steigung zu erhalten:}
\frac{\p U}{\p t} &= \frac{i}{C} = \frac{\Delta U}{\Delta t} = \frac{\unit[1]{pA}}{\unit[57,5]{pF}} = \unit[17]{mv/s} 
\intertext{Für den Isolationswiderstand gilt:}
U(t) &= U_a \cdot e^{- \frac{t}{\tau}}
\intertext{Wir differenzieren wieder:}
\frac{\p U}{\p t} = U_a \cdot \frac{1}{\tau} \cdot e^{- \frac{t}{\tau}} = - \frac{U_a}{\tau}
\intertext{Eingesetzt ergibt sich:}
- \frac{U_a}{\tau} = \frac{\unit{10}[V]}{\unit[57,5]{s}} = \unit[174]{mV/s}
\end{align*}






\end{document}